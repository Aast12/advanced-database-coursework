\documentclass[]{article}
\usepackage[margin={1in}]{geometry}
\usepackage[spanish]{babel}
\usepackage{datetime}
\usepackage{amsmath}
\usepackage{listings}
\usepackage{verbatim}
\usepackage{fancyhdr}
\usepackage{enumerate}
\usepackage[obeyspaces]{url}
\usepackage{graphicx}
\graphicspath{ {.} }
\newdateformat{hwdate}{\THEDAY~de \monthname[\THEMONTH], \THEYEAR}
\newcommand{\class}{TC3041. Base de Datos Avanzadas}
\newcommand{\hwname}{Reglas en modelo Objeto Relacional}

\pagestyle{fancy}
\fancyhf{}
\lhead{\class \\ \hwname}
\rhead{Andrés Alam Sánchez Torres (A00824854) \\ \hwdate\today}

\lstset{breaklines=true,
        language=SQL,
        extendedchars=true,
        basicstyle=\ttfamily\small,
        showstringspaces=false,
        literate={é}{{\'e}}1 {É}{{\'E}}1 
        } 

\begin{document}
    \setlength{\headheight}{23.10004pt}
    \addtolength{\topmargin}{-11.10004pt}    

    \noindent
    A la base de datos universitaria (definidaen ejercicio anterior.) se le deben agregar las 
    siguientes restricciones y/o reglas de operación (o reglas de negocio). Especifique las 
    reglas o restricciones que se deben de agregar. Suponga cada caso independiente.

    \begin{enumerate}[a)]
        \item Los profesores deben ganar un salario mínimo de \$1000.00 dls al mes.
        
        \begin{lstlisting}
            create rule min_salario as
            on update to Empleado.salario_por_hora 
                where new.salario_por_hora * 160 < 1000
            do update Profesor
                set salario_por_hora = 1000 / 160
                where Profesor.rfc = current.rfc
        \end{lstlisting}

        \item Todo director de departamento debe ser también un profesor.
        
        No se valida con una regla, sino con la definición de la base de datos (El director es una 
            referencia a un profesor).

        \item Dos o más cursos no deben tener asignado el mismo salón el mismo día a la misma 
            hora.
        
        \begin{lstlisting}
            create rule uniq_horario as
            on update to Horario.salon 
                where 
                    (select count(*) 
                    from horario
                    where salon = new.salon
                    group by hora, dia) > 1
            do after
                delete Horario
                where
                    Horario.dia = new.dia and
                    Horario.hora = new.hora and
                    Horario.seccion = new.seccion and
                    Horario.salon = new.salon
        \end{lstlisting}

        \item Cuando un profesor obtiene un aumento de sueldo, el director del departamento al que 
            pertenece ese profesor también obtiene un aumento salarial en al menos la misma cantidad.

            \begin{lstlisting}
                create rule salario_dir as
                on update to Profesor.salario_por_hora
                    where
                        new.deref(trabajan_para).deref(dirige).salario_por hora < new.salario_por_hora
                do 
                    update Profesor
                    set salario_por_hora = new.salario_por_hora
                    where rfc = new.deref(trabajan_para).deref(dirige).rfc
            \end{lstlisting}

        \item Cada vez que un profesor recibe un aumento de sueldo, el director del departamento al 
        que pertenece ese profesor recibe un aumento salarial en almenos la misma cantidad. Además, 
        cuando un profesor recibe un aumento salarial el presupuesto del departamento debe incrementarse 
        para ser mayor a la suma de los sueldos detodos los profesoresen el departamento (pude ser 
        necesario agregar el atributo presupuesto a depto)

        \begin{lstlisting}
            create rule salario_dir as
            on update to Profesor.salario_por_hora
                where
                    new.deref(trabajan_para).deref(dirige).salario_por hora < new.salario_por_hora
            do 
                update Profesor
                set salario_por_hora = new.salario_por_hora
                where rfc = new.deref(trabajan_para).deref(dirige).rfc

            create rule presupuesto as
            on update to Profesor.salario_por_hora
                where
                    new.salario_por hora > new.deref(trabajan_para).presupuesto
            do 
                update Departamento
                set presupuesto = new.salario_por_hora
                where nombre_dept = new.deref(trabajan_para).nombre_dept
        \end{lstlisting}
    \end{enumerate}
    
\end{document}